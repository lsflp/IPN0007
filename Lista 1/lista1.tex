\documentclass{article}
\usepackage[utf8]{inputenc}
\usepackage[brazil]{babel}
\usepackage{url}
\usepackage{tabularx}
\usepackage{palatino}
\usepackage{hyperref}
\usepackage[margin=2.5cm]{geometry}

\title{\textbf{Lista 1 de IPN0007 - Redes Neurais na Engenharia Nuclear}}
\author{
    \textbf{Aluno:} Luís Felipe de Melo  \\
    \textbf{Número USP:} 9297961
    }
\date{}
\begin{document}
\maketitle

\section*{Exercício 1}

Vida Artificial (VA) é um conceito que descreve os estudos que procuram modelos computacionais para problemas biológicos e tentam criar soluções baseadas na biologia para problemas de engenharia. As técnicas de VA são inspiradas em como os organismos vivos funcionam.

A Inteligência Artificial (IA) também se inspira na vida para suas técnicas. No entanto, o foco é na inteligência, propriamente dita, procurando técnicas de raciocínio, por exemplo. O principal modelo seguido na IA é o cérebro.

\section*{Exercício 2}

As Redes Neurais Artificiais (RNA) são sistemas de aprendizado de máquina baseados nos neurônios humanos. Assim como seu equivalente biológico, as redes neurais são compostas de unidades de processamento simples que são combinadas em camadas.

Como mencionado acima, sua origem é a neurociência, que estuda o sistema nervoso humano. O objetivo primário era de modelar redes neurais reais, mas com a evolução dos estudos da área de IA, elas vêm sendo aplicadas para resolver problemas do mundo real.

\section*{Exercício 3}
\section*{Exercício 4}
\section*{Exercício 5}
\section*{Exercício 6}
\section*{Exercício 7}
\section*{Exercício 8}
\section*{Exercício 9}
\section*{Exercício 10}

\end{document}
